%!TEX root = ../main.tex

\chapter{Aktuelle Version}
Die aktuelle Version der Platine wurde im Jahr 2014 entwickelt und ist seitdem bei verschiedenen Projekten in Betrieb.
Sie dient als Vorlage und Orientierung für die Neuentwicklung.

\section{Aufbau}
Als Prozessor wird ein MB9BF524KPMC von Fujitsu verwendet, dabei handelt es sich um einen 32-Bit ARM Cortex-M3 Mikrocontroller.
Ebenfalls von Fujitsu stammt der 8K x 8 Bit \ac{FRAM} Speicher mb85rs64v, der über ein \ac{SPI} mit dem Mikrocontroller
verbunden ist. Als externe \ac{RTC} wird das Modell PCF2123TS von NXP eingesetzt, welches wiederum mit einem weiterem \ac{SPI}
angeschlossen wird.\\
Die Platine verfügt über verschiedene Stecker und Buchsen, um mit diversen Externen Komponenten verbunden zu werden:
\begin{itemize}
    \item 1\(\times\) Mini-USB-B Buchse (nicht Bestückt)
    \item 4\(\times\) 2 Pol StdMol Buchse
          \begin{itemize}
              \item 1\(\times\) 1Wire
              \item 1\(\times\) Magnet
              \item 1\(\times\) Kondensatoren
              \item 1\(\times\) Automaten Stromversorgung
          \end{itemize}
    \item 1\(\times\) 3 Pol StdMol Buchse für LEDs
    \item 1\(\times\) RJ45 Buchse oder 1\(\times\) 3 Pol StdMol Buchse für RS232
    \item 1\(\times\) Steckerleiste 2\(\times\)6 Pins (Kombo-Stecker)
    \item 1\(\times\) Steckerleiste 2\(\times\)5 Pins (Debug Stecker)
\end{itemize}
Die Stromversorgung erfolgt über einen Boost Wandler, der aus der 24V Automaten Spannung eine 12V Spannung erzeugt. Mit diesen 12V
werden dann die Magneten und ein 3,3V Linearregler versorgt, welcher die Versorgungsspannung für den Mikrocontroller bereitstellt.
Drei 4700µF Kondensatoren, die an den 12V angeschlossen sind dienen als Puffer für die Magneten, wenn das Schloss ohne interne 
Spannungsversorgung betrieben wird.

\section{Funktionsweise}
Der Automat und das E-Schloss kommunizieren über eine RS232 Schnittstelle miteinander. Die Schnittstelle zum Benutzer ist
ein iButton Reader, der sich an der außenseite des Automaten befindet. \\
Das E-Schloss kann entweder mit der Automaten internen 24V Schiene betrieben werden oder bei Ausfall dieser mit einer externen
Batterie kurzzeitig über das OneWire Interface aufgeladen werden. \\
Wenn ein iButton an den Reader gehalten wird, wird die ID des iButtons an den Automaten gesendet. Dieser prüft, ob diese ID
autorisiert ist und sendet dann ein Signal an das E-Schloss. Dieses hebt dann die Magneten und öffnet damit für eine
einstellbare Zeit die Tür.

\section{Probleme}
Die Platine ist voll funktionsfähig und hat sich in der Praxis bewährt. Es gibt jedoch einige Punkte, die bei einer
Neuentwicklung verbessert werden sollten. \\
Der Grund für eine Neuauslegung der Platine ist das Ende des Produktlebenszyklus des verwendeten Mikrocontrollers.
Dieser wird nicht mehr hergestellt und ist nur noch schwer zu beschaffen. \\
Außerdem ist die Platine nicht mehr auf dem neusten Stand der Technik. So wird zum Beispiel heute eher \ac{CAN} als RS232
zur Kommunikation mit dem Automaten verwendet. In der langjährigen Benutzung ist aufgefallen, dass ein etwas größerer
Energiespeicher für das E-Schloss wünschenswert wäre.