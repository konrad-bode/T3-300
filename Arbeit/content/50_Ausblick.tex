%!TEX root = ../main.tex

\chapter{Zusammenfassung}
% Aufgabenstellung, Vorgehensweise und wesentliche Ergebnisse werden kurz und präzise dargestellt und kritisch reflektiert. Die Zusammenfassung ist eigenständig verständlich. Länge ca. 1 bis 1,5 Seiten (Problem, Ziele, Vorgehensweise, Ergebnisse und Ausblick).

Mit dieser Projektarbeit ist eine neue Platine für das E-Schloss entstanden, die den aktuellen Stand der Technik widerspiegelt. \\
Nachdem die aktuelle Version nicht mehr hergestellt werden kann war es notwendig, eine neue Version zu entwickeln welche
die gleichen Funktionen bietet und somit rückwärtskompatibel ist. Bei dieser Gelegenheit ist es zusätzlich sinnvoll einige
Punkte zu verbessern, die durch den Betrieb der Version 1 aufgefallen sind. 

Bei dem E-Schloss handelt es sich um eine Baugruppe die in den Fahrscheinautomaten der Firma krauth technology GmbH verbaut wird.
Sie dient dazu die Tür des Automaten zu öffnen wenn ein berechtigter Benutzer einen Chip mit passender Signatur an den Reader
hält. Die Platine dieses E-Schlosses öffnet im Anschluss die Tür indem der Magnet angehoben wird, der die Tür geschlossen hält. 
Um die Tür auch bei einem Stromausfall öffnen zu können, werden interne Kondensatoren durch eine externe Batterie aufgeladen
und über deren Energie dann das E-Schloss betrieben.

Als erstes mussten die Anforderungen an die neue Platine definiert werden. Dazu gehören die Maße der Platine, die Anschlüsse
und die Funktionen. Um die erwünschte Rückwärtskompatibilität zu gewährleisten, müssen die Maße und Anschlüsse der alten Platine 
übernommen werden. Die Anforderungen, die nicht durch die alte Platine vorgegeben sind, wurden durch die Erfahrungen aus dem
Betrieb der alten Platine definiert. 

Mit diesen Anforderungen kann dazu übergegangen werden die neue Platine zu entwerfen. Dazu wurden als erstes die 
Komponenten ausgewählt, die auf der Platine verbaut werden sollen. Dabei wurde darauf geachtet, dass die Bauteile
möglichst wenig Strom verbrauchen, um die Batterie zu schonen. Außerdem wurde darauf geachtet, sie so zu wählen, dass
sie auch in Zukunft noch verfügbar sind.

\

Mit den ausgewählten Bauteilen wurde dann der Schaltplan entworfen. Bei einigen Schaltungen wurde auf die Schaltungen
der alten Platine zurückgegriffen, bei anderen wurde eine neue Lösung entwickelt. \\
Besonders aufwändig waren dabei die Schaltungen der beiden Spannungswandler. Diese wurden mit Hilfe von Datenblättern
und Referenzschaltungen entwickelt und anschließend mit Hilfe von Simulationen optimiert, bis sie die gewünschten
Eigenschaften aufwiesen. 

Nachdem der Schaltplan korrekt entworfen wurde, konnte daraus ein Layout erstellt werden. Dabei wurde darauf geachtet,
dass die Platine möglichst kompakt ist und die Bauteile möglichst effizient angeordnet sind. Außerdem war die 
Störungssicherheit ein wichtiger Faktor, der bei vielen Entscheidungen berücksichtigt werden musste.

Mit diesem Schritt endet der Umfang dieser Projektarbeit. Da die Platine jedoch noch nicht vollkommen fertig 
entwickelt ist, hier noch ein Ausblick auf die weiteren Entwicklungsschritte bis das E-Schloss V2 in Serie genommen
werden kann. 

Der nächste Schritt in der Entwicklung ist dann die Fertigung der Platine, die Bestückung und die Inbetriebnahme.
Dabei wird das \ac{pcb} von einer spezialisierten Firma gefertigt und die Bauteile werden nach und nach aufgelötet
wobei immer wieder die Funktion überprüft wird. \\
Zu der Inbetriebnahme gehört die Programmierung des Mikrocontrollers und die Überprüfung der Funktionen. Dabei
wird die Platine an einen Testaufbau angeschlossen, der die Funktionen des E-Schlosses simuliert. \\
Nachdem die Funktion der Platine mit einer Inbetriebnahme Software überprüft wurde, kann die finale Software
aus der alten Version entwickelt werden, indem die alte Software an die neue Hardware angepasst wird. 