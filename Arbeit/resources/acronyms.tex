%!TEX root = ../main.tex

\addchap{Abkürzungsverzeichnis}

\begin{acronym} [MOSFET]               % längste Abkürzung
	% \setlength{\itemsep}{-\parsep}     % Kein Abstand, kompakte Darstellung
	% Sortiere von Hand oder automatisch mit Kommandozeile (windows): sort file.txt /O file.txt

	\acro{ADC} 		{Analog-Digital-Converter}
	\acro{CAN} 		{Controller Area Network}
	\acro{CPU} 		{Central Processing Unit}
	\acro{CSIO} 	{Clock Synchronous Serial Interface}
	\acro{DCE} 		{Data Communication Equipment}
	\acro{DHBW} 	{Dualen Hochschule Baden-Württemberg}
	\acro{DRSS} 	{Dual Random Spread Spectrum}
	\acro{DTE} 		{Data Terminal Equipment}
	\acro{ECU} 		{Electronic Control Unit}
	\acro{EMI} 		{Electromagnetic Interference}
	\acro{ESD} 		{Electrostatic Discharge}
	\acro{FRAM} 	{Ferroelectric Random Access Memory}
	\acro{GPIO} 	{General Purpose Input/Output}
	\acro{IC} 		{Integrated Circuit}
	\acro{I2C} 		{Inter-Integrated Circuit}
	\acro{JTAG} 	{Joint Test Action Group}
	\acro{LIN} 		{Local Interconnect Network}
	\acro{LED}		{Light Emitting Diode}
	\acro{MOSFET} 	{Metal Oxide Semiconductor Field Effect Transistor}
	\acro{pcb} 		{printed circuit board}
	\acro{PWM} 		{Pulsweitenmodulation}
	\acro{RAM} 		{Random Access Memory}
	\acro{ROM} 		{Read Only Memory}
	\acro{RTC} 		{Real Time Clock}
	\acro{SEPIC} 	{Single-ended primary-inductor converter}
	\acro{SPI} 		{Serial Peripheral Interface}
	\acro{SWD} 		{Serial Wire Debug}
	\acro{TTL} 		{Transistor-Transistor-Logik}
	\acro{TVS} 		{Transient Voltage Suppression}
	\acro{UART} 	{Universal Asynchronous Receiver Transmitter}
	\acro{USART} 	{Universal Synchronous/Asynchronous Receiver Transmitter}
	\acro{USB} 		{Universal Serial Bus}
\end{acronym}
